%preamble document set-up

%-----------------------------------------------------

%LOAD PACKAGES
%\usepackage{geometry}
\usepackage{graphicx}
%Inclusion of graphics, including resizing etc.
%\usepackage[top=2.5cm,bottom=2cm]{geometry}

% Load the geometry package with specific margin settings
\usepackage[
a4paper,
left=0.5in,
right=0.5in,
top=0.5in,
bottom=1.5in % Larger bottom margin
]{geometry}

%\geometry{a4paper, margin=0.5in}
%For page layout, etc.

%\usepackage{amssymb}

%Symbols
\usepackage{setspace} 
%Set linespacing

%\usepackage[hidelinks]{hyperref} 
%Clickable links, i.e. \ref

\usepackage[toc,nonumberlist]{glossaries} 

%Glossary Package
\makeglossary 

%\usepackage[style=altlist]{glossaries} % Alternative style
% or
%\usepackage[nonumberlist]{glossaries} % Remove page numbers

%\printglossary[title=Special Terms,toctitle=List of Terms]

%Needed to make the glossary
\usepackage{todonotes} 
%For making to-do notes in the pdf
\usepackage{float} 
%Redefine the float, now we can use 'H' instead of 'h'
\usepackage{array} 
%Extend the array environment
\usepackage{pdfpages} 
%Allows the inclusion of external pdf's 
\usepackage{color, colortbl}
%Allows us to manage the colour of items in the final output 
\usepackage{fancyhdr}
%Allows control of the headers and footers
\usepackage{nameref}
%Refer to sections by their name
\usepackage{chemfig} 
%Draw chemical diagrams
\usepackage{listings} 
%Typeset programming languages
\lstset{language=bash, numbers=left, frame=single, breaklines=true, numberstyle=\footnotesize}
%Set-up date for listings package
\usepackage{framed} 
%Framed or shaded regions that can break across pages
\usepackage[os=win]{menukeys}
%Format menu sequences, paths and keystrokes from lists
%\usepackage{draftwatermark} 
%Put a grey textual watermark on document pages
\usepackage{tikz} 
%Used for drawing within the package
\usetikzlibrary{positioning,decorations.markings,patterns}
\usepackage{multirow} 
%Create tabular cells spanning multiple rows
\usepackage[scientific-notation=true]{siunitx} 
%For putting certain scientific notation i.e. x10^.
\usepackage{titlesec} 
%Custom titles
\usepackage[normalem]{ulem}
%Strikethrough text with \sout{}

\usepackage{booktabs}

\usepackage{ragged2e}      % For \RaggedRight command to left-align paragraph columns
\usepackage{enumitem}      % For customizing lists to be more compact in tables

\usepackage{xcolor}      % Required for colors
\usepackage{fancyhdr}

%\usepackage{newtxtext} % For text font (Times New Roman-like)
%\usepackage{newtxmath} % For math font to match

%\usepackage{fontspec}
%\setmainfont{Times New Roman}

%\usepackage{tikz} % For the tinting effect
%\usepackage{lipsum} % For dummy text

%\usepackage{natbib}
%\usepackage[style=numeric, backend=biber]{biblatex}
%\addbibresource{references.bib}
%%\bibliographystyle{ieeetr} % A popular style for technical papers
%
%%\bibliographystyle{plain}  % Or 'ieeetr', 'apalike', etc.
%\bibliography{references}  % Without .bib extension

%-----------------------------------------------------
%Paragraph and Chapter Styling,

%These are the two commands that allow subsubsections to have numbering and show in the T.O.C.
%The latex numbering system is as follows:

% -1 Part
% 0 Chapter
% 1 Section
% 2 Subsection
% 3 Subsubsection
% 4 Paragraph
% 5 Subparagraph

\titleformat*{\section}{\LARGE\bfseries}
\titleformat*{\subsection}{\Large\bfseries}
\titleformat*{\subsubsection}{\large\bfseries}

%\setcounter{tocdepth}{1} % Show sections
\setcounter{tocdepth}{2} % + subsections
%\setcounter{tocdepth}{3} % + subsubsections
%\setcounter{tocdepth}{4} % + paragraphs
%\setcounter{tocdepth}{5} % + subparagraphs

% Example: Remove all space before the chapter title
%\titlespacing*{\chapter}{0pt}{0pt}{10pt}

% Example: Make the space before the chapter title negative (pulls it up)
% \titlespacing*{\chapter}{0pt}{-20pt}{40pt}

\titleformat{\chapter}[display]
{\normalfont\huge\bfseries}{\chaptertitlename\ \thechapter}{20pt}{\Huge}
\titlespacing*{\chapter}{0pt}{-5pt}{40pt} % Adjust the middle value (negative works)

%-----------------------------------------------------

%Page and Header Styling

\setlength{\headheight}{20pt} % Adjust based on your header height
\setlength{\headsep}{1in}  % Space between header and text body

%\pagestyle{fancy}
%\renewcommand{\headrulewidth}{0.25pt}
%
%\fancyhead[L]{\textbf{NAME} - \textit{ID NO}}
%\fancyhead[R]{DEPARTMENT - COMPANY}
%
%\setlength\voffset{0in}


%  DEFINE CUSTOM COLOR
\definecolor{lightgray}{gray}{0.7}

% 3. SETUP PAGE STYLE AND HEADER
\pagestyle{fancy}
\fancyhf{} % Clear default header/footer

\renewcommand{\headrulewidth}{0.4pt} % Set rule thickness

% Define a custom color named 'lightred' (40% red, 60% white)
%\definecolor{lightred}{named}{red!40}
\colorlet{lightred}{red!40} % Correct

% 3. CUSTOMIZE THE HEADER RULE
% Set the line's thickness
\renewcommand{\headrulewidth}{3.0pt}

% Redefine the rule command to add color
\renewcommand{\headrule}{%
	{\color{red}\hrule width\headwidth height\headrulewidth \vspace{\headruleskip}}%
}

%% Set the footer as well (optional)
%\fancyfoot[C]{\color{lightgray}\thepage}
%
%% Define the header content in a new command
%\newcommand{\makeheader}{%
%	
%	\begin{minipage}[c]{0.2\textwidth}
%		\centering
%		\includegraphics[width=\linewidth] {./graphics/raitlogo_light.png}
%%			\includegraphics[width=0.4\textwidth]{./graphics/raitlogo.jpeg}\\[1cm] % Replace with actual logo
%	\end{minipage}% <--- This '%' is important to prevent extra space
%	\hspace{0.05\textwidth}%
%	\begin{minipage}[c]{0.75\textwidth}
%		\vspace*{0.5cm} % Add some space to not overlap with the custom header
%		\centering
%		%\sffamily % Use sans-serif font
%%		\rmfamily % Use Times new roman
%		\textbf{\large RAMRAO ADIK INSTITUTE OF TECHNOLOGY, NERUL} \\[2.5ex]
%		\textbf{DEPARTMENT OF COMPUTER ENGINEERING} \\[1.5ex]
%		\textbf{\small Academic Year:  2025}
%		\vspace*{0.3cm} % Add some space to not overlap with the custom header
%	\end{minipage}%
%}
%
%% Set up the page style and place the header
%\pagestyle{fancy}
%\fancyhf{} % Clear all header and footer fields
%%\renewcommand{\headrulewidth}{0pt} % Remove the default header line
%
%\fancyhead[C]{\color{lightgray} \makeheader}
%\vspace*{1cm} % Add some space to not overlap with the custom header
%% Redefine the rule command to add color
%\renewcommand{\headrule}{%
%	{\color{lightred}\hrule width\headwidth height\headrulewidth \vspace{\headruleskip}}%
%}
%
%% --- 2. THE FIX: REDEFINE THE 'PLAIN' STYLE FOR CHAPTER PAGES ---
%\fancypagestyle{plain}{%
%	%\fancyhf{} % Clear header/footer fields for the 'plain' style
%	%\renewcommand{\headrulewidth}{0.4pt} % Add the rule to chapter pages
%	
%	% Apply the same settings to the 'plain' style
%	\fancyhead[R]{\color{lightgray} \makeheader}
%	\fancyhead[L]{\color{lightgray} \makeheader}
%%	\fancyhead[]]{\color{lightgray} \makeheader}
%	
%%	\\fancyfoot[C]{Confidential}
%}
%
%	\fancyhead[R]{\color{lightgray} \makeheader}
%	\fancyhead[L]{\color{lightgray} \makeheader}
%\fancyhead[]]{\color{lightgray} \makeheader}


% Set the footer as well (optional)
\fancyfoot[C]{\color{lightgray}\thepage}

% Define the header content in a new command
\newcommand{\makeheader}{%
	\begin{minipage}[c]{0.2\textwidth}
		\centering
		\includegraphics[width=\linewidth] {./graphics/raitlogo_light.png}
	\end{minipage}%
	\hspace{0.05\textwidth}%
	\begin{minipage}[c]{0.75\textwidth}
		\vspace*{0.5cm} % Add some space to not overlap with the custom header
		\centering
		\textbf{\large RAMRAO ADIK INSTITUTE OF TECHNOLOGY, NERUL} \\[2.5ex]
		\textbf{DEPARTMENT OF COMPUTER ENGINEERING} \\[1.5ex]
		\textbf{\small Academic Year:  2025}
		\vspace*{0.3cm} % Add some space to not overlap with the custom header
	\end{minipage}%
}

% Set up the page style and place the header
\pagestyle{fancy}
\fancyhf{} % Clear all header and footer fields
%\renewcommand{\headrulewidth}{0pt} % Remove the default header line if you want no rule by default

\fancyhead[C]{\color{lightgray} \makeheader} % This applies the header to 'fancy' styled pages

% Redefine the rule command to add color
\renewcommand{\headrule}{%
	{\color{lightred}\hrule width\headwidth height\headrulewidth \vspace{\headruleskip}}%
}

% --- THE FIX: REDEFINE THE 'PLAIN' STYLE FOR CHAPTER PAGES ---
\fancypagestyle{plain}{%
	\fancyhf{} % Clear header/footer fields for the 'plain' style
	\fancyhead[C]{\color{lightgray} \makeheader} % Apply the custom header to 'plain' pages
	\fancyfoot[C]{\color{lightgray}\thepage} % Make sure the footer also appears on 'plain' pages
	\renewcommand{\headrule}{% % Ensure the header rule is also present for 'plain' pages
		{\color{lightred}\hrule width\headwidth height\headrulewidth \vspace{\headruleskip}}%
	}
}

%-----------------------------------------------------

%Watermark
%\SetWatermarkText{DRAFT}
%\SetWatermarkScale{4}
%\SetWatermarkLightness{0.8}
%\SetWatermarkColor{red}

%-----------------------------------------------------

%Document meta-data

%\hypersetup{
%    pdftitle={TITLE HERE},
%    pdfauthor={Author},
%    pdfsubject={Subject},
%    pdfcreator={Author},
%	pdfstartview={XYZ null null 1.00},
%}

% Hyperlink setup
\hypersetup{
	colorlinks=true,
	linkcolor=blue,
	filecolor=magenta,
	urlcolor=cyan,
	citecolor=teal,
}
%-----------------------------------------------------

%reduce pdf size

\pdfminorversion=7
\pdfobjcompresslevel=2
    
%-----------------------------------------------------

%Custom Commands

\definecolor{Grey}{gray}{0.95}
\lstset{language=bash, numbers=left, frame=single, breaklines=true, numberstyle=\footnotesize}

%-----------------------------------------------------

%Include Glossary

%\input{./supportfiles/gloss}

%-----------------------------------------------------

%Tiny Leaves

\newcounter{leafs}
 
\tikzset{%
leaf/.style={/utils/exec=\setcounter{leafs}{0},
    decorate,decoration={
        markings,
        mark=at position 0 with {
            \draw [bend left=35,fill=black] (0,0) to ++(-0.3,0.02*\sign) to ++(0.3,-0.02*\sign);},
        mark=between positions 0.02 and 0.5 step 0.05 with {
            \draw [bend right=35,fill=black] (0.01*\sign,0.01*\sign) to ++(-0.2-\mult,0.3*\sign+\mult*\sign) to ++(0.2+\mult,-0.3*\sign-\mult*\sign);
            \stepcounter{leafs};},
        mark=between positions 0.54 and 0.99 step 0.04875 with {
            \addtocounter{leafs}{-1}
            \draw [bend left=35,fill=black] (-0.01*\sign,0.01*\sign) to ++(0.2+\mult,0.3*\sign+\mult*\sign) to ++(-0.2-\mult,-0.3*\sign-\mult*\sign);},
    }
}}
 
\newcommand{\leaves}[1]{%
\setlength{\lineskip}{6pt plus 6pt minus 0pt}\lineskiplimit=\baselineskip
\def\mult{0.03*\theleafs}
\begin{tikzpicture}
    \node [align=center] (box) {\uppercase{#1}};
    \node [below left=of box] (ll) {};
    \node [above left=of box] (ul) {};
    \node [below right=of box] (lr) {};
    \node [above right=of box] (ur) {};
    \def\sign{-1}
    \draw [bend right=45, fill=black,postaction=leaf] (ul) to (ll) to ++(0.1,0.1) to [bend left=45] (ul);
    \def\sign{1}
    \draw [bend left=45, fill=black,postaction=leaf] (ur) to (lr) to ++ (-0.1,0.1) to [bend right=45] (ur);
\end{tikzpicture}
}
